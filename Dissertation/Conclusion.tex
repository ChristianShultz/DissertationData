\chapter{Concluding Remarks} \label{chap::conclusion}
We started our discussion with an overview of the Standard Model, introducing Quantum Chromodynamics as the relativistic gauge field theory describing the interaction of color charged spin=$\frac{1}{2}$ fermions, quarks, with the spin-$1$ gauge bosons mediating the strong force, gluons. We commented on the non-perturbative nature of QCD which distinguishes it from the other gauge field theories featuring in the standard model -- we don't currently know how to construct analytic solutions to the theory at low energy.  In this manner QCD is perhaps the least understood portion of the Standard Model. 

Historically, a good portion of our intuition comes from models of hadronic physics. These models were built to capture the essential effective degrees of freedom and symmetries present in the experimentally observed spectrum of hadrons. We specifically restricted ourselves to the constituent quark model in which mesons appear as quark-antiquark pairs bound in a central potential of gluonic origin. By constructing $q\bar{q}$ angular momentum eigenstates we were able to illustrate a known result, namely that the predicted spectrum of quark model states, in terms of the $J^{PC}$ quantum numbers is not exhaustive. We identified a set of quantum numbers, $J^{PC} = 0^{--}, (0,2,\ldots)^{+-}, (1,3,\ldots)^{-+}$ known as exotic which may provide hints about the role of glue in Quantum Chromodynamics. 

In order to motivate some of the discussion we also commented on the upcoming GlueX experiments sited at Jefferson Lab which aims to remedy the current lack of photoproduction data in the light quark sector. One of the physics goals of this experiment is to conduct a search for the $J^{PC} =1^{-+}$ exotic states. Currently there is some tentative evidence for three candidate isovector exotic states, $\pi_1(1400), \pi_1(1600), \pi_1(2015)$, though none are without controversy. 

QCD is non-perturbative in this regime; the lattice presents a unique opportunity to provide theoretical input into the expected rate of photo production of exotic mesons. As a first step then we must develop techniques, necessary to extract radiative transition matrix elements directly from the lattice. In the text we show that these matrix elements are embedded in three-point correlation functions, which, in a general sense contain information about many transitions simultaneously. 

Our goal was then to develop analysis methods which allow us to extract a single radiative transition matrix element from a tower of transitions occurring simultaneously. We demonstrated a technique, the construction of optimized interpolating fields, which allow us to project a single contribution out of three point functions, enabling us to study both ground and excited state matrix elements directly on the lattice.  

We then proceeded to use the technique to extract formfactors and transition matrix elements for the lightest few isovector pseudoscalar and vector states in a version of QCD where there are three flavors of quarks all tuned to approximately the physical strange quark mass. Having shown the efficacy of these techniques a natural question is their range of applicability -- where else can we use these methods to study non-perturbative physics?   

One area of immediate interest is in the study of the exotic mesons which motivated a good portion of this work. Here it is a straightforward application of the methods outlined in this manuscript and one could hope to provide some theoretical input about the size of exotic photo couplings in the near term future. 

The techniques outlined can also be generalized quite readily to the baryon sector. In this case one might be interested in reactions such as $N^* \rightarrow N^* \gamma$ as a method to study the internal quark structure of excited baryons non-perturbatively. Also of interest are transitions involving nucleons, for example, $N^*\rightarrow N \gamma$ where the dependence of the transition form factor on the photon virtuality can be measured quite directly in electroproduction experiments. 

From a more theoretical perspective, these techniques are also quite interesting when applied to unstable particles. Throughout this analysis we have been working under the assumption that our states are stable. In general this is not always true, for example the $\rho$ meson occurs as a dynamically generated resonance in $\pi\pi$ scattering. On the lattice this introduces additional complications associated with form factor extraction. To date, however, there is no calculation exploring the coupling of a resonance to external currents. A rigorous calculation at physical kinematics, where the $\rho$ is a resonance, seeking the coupling $\rho\rightarrow\pi\gamma$ would in fact need to determine the P-wave partial-wave amplitude for $\pi\pi \rightarrow \pi \gamma$ as a function of the invariant mass, $m_{\pi\pi}$. By analytically continuing the amplitude to complex values of $m^2_{\pi\pi}$ and extrapolating to the $\rho$-resonance pole, the coupling could be extracted as the residue of the amplitude. Very recently \cite{Briceno:2014uqa} the formalism relating matrix elements extracted in a finite volume to the physical amplitude has been developed. Calculations aiming to extract the value of the  $\pi\pi \rightarrow \pi \gamma$ at the $\rho$-meson pole using the techniques outlined in this analysis are currently underway.

In short, the techniques laid out in this dissertation, allowing for the extraction of single matrix elements for each state in a tower of discrete eigenstates, are required for any attempt to determine excited state or resonance couplings to external currents aiming to probe these states. The technical formalism has now been implemented and explored; more complicated three-point function calculations can now be attempted. 



