\tolerance=10000

\documentclass[12pt]{report}

%This package is built on the LaTeX2e report class, so any other packages which are
%also compatible with it can also be used in combination with it.  ODUthesis should
%be in the directory in which you are working, or in one of the standard input directories
%for the TeX installation on your system.

\usepackage{ODUthesis}

%This package causes the first line of a section to be indented as required by the dissertation guide.
\usepackage{color}
\usepackage{indentfirst}

%%%%%Other LaTeX2e packages such as the AMS fonts and graphics packages can be included here.

%This style follows conventions used in Physical Review C. The names of figures, tables and captions can
%be changed by using the commands below with appropriate changes. For example, "FIG." could be replaced by
%"Fig." or "Figure" to change the labeling of figure captions.

%\renewcommand{\figurename}{FIG.}
%\renewcommand{\tablename}{TABLE}
%\renewcommand{\bibname}{BIBLIOGRAPHY}



\begin{document}

\title{Thesis/Dissertation title goes here}

\author{Author's name goes here}
\principaladviser{advisor's name goes here}
\member{member 1}  %This command produces a signature line for the specified member on the
\member{member 2}  %title page. Use on instance of this command for each member of the
                   %committee other than the advisor up to a total of 5 members.

\degrees{description of previous degrees goes here}

\dept{department name goes here}          %for example \dept{physics}

\submitdate{month that the degree is to be awarded}

%\phdfalse          %produces language on title page for Masters Thesis. Otherwise the default
                    %is for a Ph.D. dissertation.

%\copyrightfalse    %suppresses copyright notice

%\figurespagefalse  %suppresses List of Figures

%\tablespagefalse   %suppresses List of Tables

\vita{The text of the Vita goes here.}


\abstract{text of abstract goes here}

\beforepreface

\prefacesection{Acknowledgements}

    %text of Acknowledgements goes here
    %any other preface section is inserted similarly by the command
    %\prefacesection{Section Title}

\afterpreface

    %The text of the thesis/dissertation begins here. The basic organization is
    %in chapters, sections, subsections. 

\chapter{}

\section{}

\subsection{}

\begin{figure}
\caption{bar}
\end{figure}


  %This begins the list of reference or bibliography. The default form is consistent
  %with the references in the Physical Review. The articles must be listed in the order
  %in which they appear in the text. The style of the bibliography can be changed and
  %automatic ordering of the entries can be accomplished using BibTeX. Use of BibTeX is
  %explained in most of the standard LaTeX books.

\begin{thebibliography}{99}

\addtocontents{toc}{\vspace*{12pt}}  %This command adds some extra space in the table of
                                     %contents
\addcontentsline{toc}{chapter}{BIBLIOGRAPHY}  %This command adds and entry for the
                                              %bibliography in the table of contents

\bibitem{brownjackson-ref}  G.~E.~Brown and A.~D.~Jackson, {\it The
Nucleon--Nucleon Interaction} (North--Holland, Amsterdam, 1976).

\end{thebibliography}

%This command begins the Appendix section. The style of the chapter numbering is changed to
%letters

\appendix

\chapter{}



\newpage

%The command below initiates printing of the vita page.  The name and other information is taken
%from previous entries.

\vitapage


\end{document}
