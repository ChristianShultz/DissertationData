\chapter{Introduction} \label{chap::intro}

%{\color{red}
%Goals:
%\begin{enumerate}
%\item I want to get to the idea that photo-couplings are a way of probing exotic hadronic matter, I need to: 
%\begin{itemize} 
%\item Tell people why they should care
%\item define exotic hadron matter
%\end{itemize}
%\end{enumerate}
%} % red

%{\color{blue}
%Why should we care?
%}

Our modern understanding of particle physics is built upon quantum field theories possessing local gauge invariance which describe the interactions of fundamental spin-$\frac{1}{2}$ fermions with spin-$1$ gauge bosons. The standard model is built out of two such theories, Electroweak theory\footnote{Electroweak theory is the unified theory of Quantum Electrodynamics and Weak Theory. Spontaneous symmetry breaking causes $U(1)$ and the `z-projection' of $SU(2)$ to mix, one linear combination becoming the photon of Quantum Electrodynamics, the orthogonal combination the $Z$-boson.} - in which the gauge group is $U(1)\times SU(2)$, and Quantum Chromodynamics, which possesses local $SU(3)$ gauge invariance. 

For orientation we first consider the more familiar case of Quantum Electrodynamics. The Lagrangian for this theory, describing the interaction spin-$\frac{1}{2}$ fermions via exchange of gauge bosons (photons), is 
\begin{equation*}
\mathcal{L} = \psibar \left( i\slashed{\partial} - m \right) \psi +gA^\mu \bar{\psi} \gamma_\mu \psi - \frac{1}{4} F^{\mu\nu } F_{\mu\nu}.
\end{equation*}
The piece $\psibar \left( i\slashed{\partial} - m \right) \psi$ is the Lagrangian for a free spin-$\frac{1}{2}$ particle which yields the Dirac equation. The second piece, $gA^\mu \bar{\psi} \gamma_\mu \psi$, describes the coupling of the gauge field to the electrically charged vector current, while the term $F^{\mu\nu}F_{\mu\nu}$, featuring the field strength tensor $F_{\mu\nu}$, contains the kinetic term for the gauge bosons\footnote{In a  non-abelian theory $F_{\mu\nu}$ also contains the self-couplings of the gauge bosons.}.

In this theory the coupling, $g$, is small and one can perform perturbative expansions. This is to say that one can calculate observables, in a systematically improvable manner, via performing an expansion in the coupling. Such a theory, in which there is a small parameter in which to expand, is called \emph{perturbative}. 

The Weak theory is mediated by three gauge bosons commonly referred to as the charged and neutral currents corresponding to the $W^{\pm}$ and $Z$ gauge bosons. Here the gauge bosons are massive, $m_{Z} \sim 90 \mathrm{GeV}$, $m_{W^\pm} \sim 80 \mathrm{GeV}$, and at low energies one can perform a perturbative expansion in $\frac{p^2}{m^2}$ where $p$ and $m$ are the momentum and mass of the gauge boson. We can construct an \emph{effective field theory}, describing the low energy dynamics of the weak theory, which is perturbative\footnote{This is possible due to the large masses associated with the W and Z bosons. }. 

Quantum Chromodynamics, at low energies, is not perturbative. Like QED and the Weak Theory it is a relativistic gauge theory. The theory corresponds to the $SU(3)$ piece of the $SU(3)\times SU(2) \times U(1)$ Standard Model of particle physics and describes the interaction of color charged fermions, quarks, with gauge bosons, the gluons. A Lagrangian with local $SU(3)$ gauge invariance for the theory can be obtained in the standard manner by `gauging' the derivative and including the lowest dimensional gauge and parity invariant function of the field strength tensor
\begin{equation*}
\mathcal{L}_{QCD} = \psibar \left( i\slashed{\partial} - m \right) \psi +gA^\mu \bar{\psi} \gamma_\mu \psi - \frac{1}{4} G^{\mu\nu } G_{\mu\nu}.
\end{equation*}

%As in the preceding discussion, $\psibar \left( i\slashed{\partial} - m \right) \psi$ is the Lagrangian for a free spin-$\frac{1}{2}$ particle which yields the Dirac equation. 
In QCD, the piece $gA^\mu \bar{\psi} \gamma_\mu \psi$ describes the coupling of the gauge field to the \emph{color} charged vector current, while the term $G_{\mu\nu}$ is the field strength tensor. Owing to the non-abelian\footnote{Non-abelian refers to the fact that generators of the gauge group do not commute. A more familiar physical example is the rotation group, $SO(3)$. Rotating an object $\pi/2$ away followed by a clockwise rotation of $\pi/2$ will not yield the same result as the opposite ordering. This is because the generators of the rotation group do not commute $[J_i,J_j] = \epsilon_{ijk} J_k$. In QCD the gauge group is $SU(3)$; there are matrices at every point in space-time which do not commute. In order to properly account for the color phase we must specify the endpoints and the \emph{path} between them. This is to be contrasted with QED which is an example of an abelian theory.}
 nature of the theory the gauge portion of the Lagrangian, $G^{\mu\nu } G_{\mu\nu}$, also includes gluon-gluon interactions. The field strength tensor can be obtained as the commutator of two gauge covariant derivatives\footnote{The gauge covariant derivative $D_\mu$ is defined as $D_\mu = \partial_\mu - i g A_\mu$.}, 
\begin{equation*}
G_{\mu\nu} = ig^{-1} \left[ D_\mu , D_\nu \right] = \partial_\mu A_\nu - \partial_\nu A_\mu -ig\left[A_\mu, A_\nu\right].
\end{equation*}
From which we can see that the square of this term will contain three and four field combinations which generate the three and four gluon vertices present in QCD. 

The fermion fields featured in $\mathcal{L}_{QCD}$ come in six \emph{flavors}, up, down, charm, strange, top, and bottom. Each quark has a different mass, $u$ and $d$, quarks corresponding to the up and down flavors are quite light\footnote{Quark masses are renormalization scheme and scale dependent quantities. The hierarchy of masses is presented to give the reader an intuitive notion of the range of quark masses featuring in QCD.}, $m_{u,d} \sim \mathcal{O}(5\mathrm{MeV})$. The strange quark is a bit heavier, $m_s \sim \mathcal{O}(100\mathrm{MeV})$, while the charm, top, and bottom are much heavier. The QCD coupling is `flavor-blind' in the sense that each quark flavor couples to gluons with the same  strength, $g$. 

Owing to the approximate mass degeneracy between the up and down quark flavors QCD also possesses an approximate \emph{isospin} symmetry. The gauge field interaction is flavor blind, if the masses of the up and down quark are exactly the same then we would expect to find exact degeneracies in the experimental spectrum corresponding to the $u\rightarrow d$ isospin symmetry\footnote{Even if the quark masses were identical the $u$ and $d$ quarks have different charges and thus the symmetry would be broken by QED interactions. }. This apparent symmetry can be seen in the degenerate masses of different charge states present in the experimental spectrum of QCD. In the meson sector, we could for example look at the different charge states of the kaon, $m_{K^\pm}  = 494 \mathrm{MeV}$, $m_{K^0,\bar{K}^0} = 498 \mathrm{MeV}$.  The $\Delta$ baryon is another example of the apparent isospin symmetry, the four different charge states, corresponding to different combinations of $u$ and $d$ quarks, all have masses at approximately $m_\Delta \sim 1232 \mathrm{MeV}$. 

Another interesting feature of QCD is the `running' of the coupling constant $g$. At high energies $g$ becomes small and perturbative calculations are successful. Indeed the \emph{3-jet} events, first observed at DESY, provide perhaps the strongest evidence of the existence of gluons and the validity of perturbative QCD at large energies\footnote{In a simple sense jets occur when quarks `hadronize'. Since quarks are only produced in pairs an additional particle is required to explain events containing an odd number of jets -- QCD indicates that this extra particle is a particularly energetic gluon radiated by one of the quarks.}. At lower energies however the situation becomes more complicated: $g$ takes large values and perturbative expansions fail to converge. One of the most pressing questions then is understanding how the color charged degrees of freedom present in the Lagrangian bind together to form the experimentally observed spectrum of color-singlet hadrons at low energy. 

To a large degree, our understanding of hadronic spectroscopy is built on phenomenological models, particularly the constituent quark model. Mesons and baryons are thought of as composite objects formed from two and three constituent quarks respectively. Here the constituent quarks can be loosely thought of as spin $\tfrac{1}{2}$ fermions which describe the effective degrees of freedom manifest in the spectrum of observed hadrons\footnote{These constituent quarks are, in the light sector, $\mathcal{O}(300-500\mathrm{MeV})$ fermions. They correspond to `dressed' versions of the $\mathcal{O}(5-100\mathrm{MeV})$ up, down, and strange quarks appearing in the Lagrangian.}. Within such a model we then identify the hadrons as aggregate combinations of constituent quarks bound in a potential of gluonic origin. Allowing for such a picture, in conjunction with the assumption that the potential is central, permits us to construct allowed angular momentum wave functions which in turn enables us to postdict a spectrum of states in terms of their $J^{PC}$ quantum numbers.

%\footnote{Throughout, we shall refer to these hadrons, described only by valence quarks as ``conventional" and loosely refer to the models which describe only conventional quarks as constituent quark models.}. 

The spectral content of these models is not however exhaustive, they have no support for \emph{glueballs}, states composed entirely out of the gauge degrees of freedom. More relevant to our analysis, they also do not feature \emph{hybrid} hadronic matter -- states that can loosely be classified as being composed of both quark and gluonic degrees of freedom. Experimental observation and theoretical investigation of this set of hybrid states is an intriguing prospect. These states probe the thus far unmanifested low-energy gauge degrees of freedom of the underlying field theory, QCD. 

A common thread amongst the hadronic spectroscopists who create these models is a desire to understand the origin of the spectrum of states. This is expected to be difficult; we believe that the theory is confining -- the quark and gluon fields, present in the Lagrangian, are hidden. In a simple sense this means that we do not observe the fundamental field content of the theory, as opposed to, say, the electron featured in Quantum Electrodynamics. Rather we have access to a set of asymptotic hadronic eigenstates, composite objects made of quarks and gluons, whose properties are not easily inferred from the underlying Lagrangian.

Mesons in particular serve as an ideal meeting ground between theory and experiment\footnote{Mesons, unlike baryons, have a charge conjugation quantum number in addition to the spin and parity quantum numbers. This third quantum number makes identifying exotic signals more straightforward.}. Experimentally the spin and parity distribution of states in conjunction with the lack of states with strangeness or isospin greater than one indicates that mesons might be described, in a minimal context, by simply coupling together a constituent quark and antiquark into an object of spin, $S=0,1$. Inclusion of orbital angular momentum allows for a prediction of multiplets of states, based on quantum mechanical angular momentum addition rules, in terms of their spin, parity, and charge conjugation quantum numbers $J^{PC}$. 

For example, considering a quark-antiquark pair in $S$-wave with $I=1$,  yields a prediction of two states, $J^{PC} = 0^{-+}$, the pion, and $J^{PC} = 1^{--}$ the rho meson. Considering instead one unit of orbital angular momentum, a $P$-wave, one predicts $J^{PC} = 1^{+-}$ and $J^{PC} = (0,1,2)^{++}$. Opening the Particle Data Group summary table for mesons one finds experimental candidates, $b_1(1235), a_0(1450), a_1(1260),$ and $a_2(1320)$, which match the expected pattern of states. The same pattern repeats if one considers a quark anti-quark pair in $D$-wave, one again finds experimental candidates matching the predicted pattern of states\footnote{There is no $\rho_2$ experimental candidate.}.

Absent however, under a naive inspection, is any sign of a gluonic contribution. Short of transforming the bare quarks into constituent quarks\footnote{By this we mean the process by which the $\mathcal{O}(5-100\mathrm{MeV})$ quarks featuring in the Lagrangian are `dressed', by QCD, to form the $\mathcal{O}(300-500\mathrm{MeV})$ effective degrees of freedom present in the spectrum.} it seems to play no role in the spectrum. This is unexpected, QCD is a \emph{strongly} coupled gauge theory. The simple $q\bar{q}$ should not be the entire story, indeed even in the absence of quarks, in pure gauge theory, gluons can interact to form bound states formed entirely from glue, \emph{glueballs} \cite{Morningstar:1999dh}. Investigation of this phenomena in the isoscalar sector is however complicated. Glueballs are expected to mix strongly with the quark anti-quark excitations thus making it difficult to extract a clear and concise picture of the role of glue in the spectrum.

The \emph{hybrid} sector provides a more promising testing ground of the hidden gluonic contributions to the spectrum of QCD. Provided the gluonic field excitation has quantum numbers other than $0^{++}$ we can generate $J^{PC}$ outside of the set allowed in a quark-antiquark picture, for example $J^{PC} = 0^{--}, 0^{+-}, 1^{-+}, 2^{+-}$. These quantum numbers are known as \emph{exotic} and are one of the best signature for hadronic physics extending beyond the constituent quark picture. To date there has been no unambiguous experimental observation of any such quantum numbers.  

Existence of exotic excitations might also suggests the existence of hybrid candidates, with conventional quantum numbers, occurring in multiplets which should be embedded in the non-exotic spectrum. Observation of these hybrid candidates, understanding their placement in the spectrum, and calculation of their expected properties provide exciting benchmarks for experimentalists and theorist alike. In fact, a good portion of the theoretical groundwork, aimed at predicting their location in the spectrum of non-exotic states, has already been performed. Hybrid candidates have been identified in a number of lattice calculations, \cite{Dudek:2007wv,Dudek:2009qf,Dudek:2011bn,Dudek:2012ag,Dudek:2013yja,Liu:2012ze}, albeit at unphysical quark masses, the observed pattern of states responding only mildly to changes in the quark mass.  Indeed it is the observation of this hybrid multiplet in lattice calculations which spawned much of the motivation for this dissertation project. 

The tool we propose to use to investigate the dynamics giving rise to the spectrum of hadrons is lattice QCD. This is a first principles numerical approach to estimating correlation functions, theoretical quantities encoding the dynamics of QCD, based on discretizing the theory on a finite grid of Euclidean space-time points. Correlation functions are evaluated over a large but finite number of field configurations providing for a systematically improvable framework within which we can obtain information about the non-perturbative dynamics of Quantum Chromodynamics. 

Utilizing the tool, lattice QCD, in conjunction with intuition, provide by the quark model, promises to be a fruitful avenue of approach when investigating the spectrum of hadrons. In this manuscript we will concern ourselves first with the spectrum, extracting the excitations of the theory in the mesonic sector which in conjunction with novel lattice techniques will allow us to speculatively identify the lowest lying hybrid supermultiplet\footnote{This is a reproduction of work already completed by colleagues \cite{Dudek:2011bn}.}.  We will then take the next step, calculating vector current matrix elements between various hadronic states in order to provide a non-perturbative estimation of their photo-couplings, a quantity relevant to experimental physicists interested in performing photo production studies such as the GlueX experiment due to begin this year. 

When the initial and final state hadrons are of the same type we speak of \emph{formfactors}. Phenomenologically these form-factors can be related to the quark charge and current distributions within the parent hadron. The photon can also induce a transition from one initial hadronic eigenstate to another, in this case we refer to the Lorentz invariant functions encoding dynamics as \emph{transition formfactors}. 

The central problem we will be solving is extracting radiative transition matrix elements from three-point functions calculated non-perturbatively using lattice QCD. These functions have the generic structure
\begin{equation*}
\langle 0 | \mathcal{O}_f(\Delta t) j^\mu(t) \mathcal{O}^\dagger_i(0) | 0 \rangle, 
\end{equation*}
where the \emph{operators}, $\mathcal{O}_{f,i}$, are color singlet constructions, built out of the fermion and gauge fields present in QCD, capable of creating eigenstates of QCD. The vector current,  $j^\mu$, couples the external photon field to the quarks, and induces the transition from the initial to final state. 

We will find that the operators featuring in the preceding equation can create \emph{all} states with the same quantum numbers at the source and sink (e.g. spin, helicity, momentum), each state propagating through time with a factor $e^{-E_\estate{n}t}$ where $E_\estate{n}$ denotes the energy of the $\estate{n}$'th state.  The problem at hand is then to extract a \emph{single} radiative transition matrix element out of one of our three-point functions where in principle we are also interested in excited states whose contributions occur as subleading exponentials. We will show that, via the construction of \emph{optimal} operators, which dominantly produce a single state, we can isolate both ground and excited state contributions to three-point correlation functions. We will then proceed to use these matrix elements to extract radiative transition form factors directly from the lattice. 

There is also a rich phenomenology related to radiative transition formfactors arising from calculations performed in the charmonium sector as well as those obtained via non-relativistic quark models. Observation of relative scales of transition amplitudes, or equivalently relative sizes of photo-couplings, allows for a model interpretation of the underlying hadron structure.   As lattice gauge theorists we are uniquely poised to provide non-perturbative, model independent theoretical input on the size of such photo-couplings.  This manuscript will describe a calculational scheme in which one can extract these parameters from lattice calculations before implementing the method and examining the phenomenology of the results. 


%Of particular note is the notion of heavy quark spin flip suppression -- within a non-relativistic quark model this is the observation that transitions which alter the spin wave function (moving from triplet to singlet or vice versa) are suppressed by a power of the quark mass. 




%distributions of quark charge and currents within the hadron of interest. 

%GlueX proposes to search for exotic mesons via providing photo-production data, with unprecedented levels of statistics. There is some theoretical suggestion\footnote{something about the charm calculation implies they are not small up there} that these exotic an hybrid mesons are preferentially produced via photo-production mechanisms.  As lattice gauge theorists we are uniquely poised to provide theoretical input on the size of such photo-couplings. Previous lattice calculations, in the charmonium sector, indicated that the couplings were not small. The focus of this manuscript will be to adapt and improve the techniques, applying them to the light quark sector and thereby making theoretical predictions, arising from first principles calculations, of manifest experimental relevance.  

% The manuscript itself is structured as follows. 
 
% boiler plate stuff about chapter content 
